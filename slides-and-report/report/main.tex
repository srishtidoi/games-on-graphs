\documentclass[11pt, A4 paper, twocolumn ]{article}
\usepackage{palatino}
\usepackage{url}
\usepackage{latexsym}
\usepackage{lipsum}
\usepackage{titlesec}


\setlength{\paperwidth}{21cm}   % A4
\setlength{\paperheight}{29.7cm}% A4
\setlength\topmargin{-0.5cm}    
\setlength\oddsidemargin{-1cm}   
\setlength\textheight{24.7cm} 
\setlength\textwidth{18.0cm}
\setlength\columnsep{0.8cm}  
\newlength\titlebox 
\setlength\titlebox{5cm}
\setlength\headheight{1pt}   
\setlength\headsep{30pt}

\usepackage{fancyhdr}
\usepackage{hyperref}
\usepackage{graphicx}
%\usepackage{csc}% unknown package
\pagestyle{fancy}
\fancyhead[L]{Project Report (PHY 302)}
\fancyhead[R]{Srishti Patil}
\fancypagestyle{firstpage}{%
	\lhead{Project Report (PHY 302)}
	\rhead{IISER-Pune}
}
%\thispagestyle{empty}        
%\pagestyle{empty}


\title{{\Huge Evolution of Cooperation Among Autonomous Agents in the Presence of Public Information} }
\author{Srishti Patil\\ Supervisor: Dr. M.S. Santhanam}
\date{January, 2021}


\begin{document}
	\maketitle
	\thispagestyle{firstpage}
	\section{Introduction}
	\par
	In games like Prisoner's Dilemma and Public Goods Game, defection is theoretically the dominant strategy. Thus, in the classical repeated Prisoner's Dilemma game in a well-mixed population, it is hard to sustain cooperation, which is often at odds with reality where cooperation is ubiquitous. Understanding the emergence and persistence of this altruistic behaviour in this setting poses an interesting challenge. Evolutionary Game Theory provides a mathematical framework to explore this problem. Previous work on the topic has focused on spatial reciprocity and nearest neighbour interactions among agents. The relevant results have been summarized in [figure 1].  In other related works, incentivisation, such as social rewarding and punishment, is an effective method to promote cooperation.\par
	The aim of this project is to understand the effect of public information and reputation in finitely repeated spatial games (mainly Prisoner's Dilemma). In [section 2], three new models are proposed. Results are shown and discussed in [section 3]. [Section 4] explores in brief, an application of one of the models on a real-world problem.\\
	(\textbf{insert fig 1}) \\
	
	\section{Models}
	\par
	Three models have been introduces as mechanisms to understand and enhance cooperation among agents playing the Prisoner's Dilemma game on different network topologies (Regular Lattice, Scalefree, Small World and Random Regular). 
	
	\subsection{Probability Distribution Model}
	\subsection{fc-Threshold Model}
	\subsection{Reputation Model}
	\lipsum
\end{document}